\section{Resolução Questão 4}

\begin{frame}{Questão 4}
Considere que $X_1,\dots,X_n$ consiste em $n$ observações de uma amostra aleatória com distribuição estatística contínua uniforme no intervalo $[0,6]$. Considere a v.a.\ $\bar X$ derivada das v.a.’s $X_1,\dots,X_n$ da seguinte forma:
\[
\bar X \;=\;\frac{x_1 + \cdots + x_n}{n}.
\]
\end{frame}

\begin{frame}[fragile]{a) Geração das amostras}
\textbf{a)} Gere um conjunto de 1200 amostras das v.a.’s $X_1,\dots,X_n$ com $n=800$ observações cada. Para cada uma das amostras geradas anteriormente calcule os valores de  $\bar X$.

\vspace{0.5em}
\textit{Por que?} Simular as médias amostrais de $U(0,6)$ para observar seu comportamento e fundamentar os cálculos de intervalo de confiança.

\vspace{0.5em}
\lstinputlisting[
  language=Scilab,
  basicstyle=\ttfamily\scriptsize,
  caption={Geração das amostras e cálculo de $\bar X$}
]{../q04/questao04_amostras_ic.sci}
\end{frame}

% Frame 1: Contexto e Normalização
\begin{frame}{b) Intervalos de Confiança (1/3)}
\textbf{b)} Para cada amostra das v.a.’s $X_1,\dots,X_n$ geradas no item anterior, construa o intervalo de confiança para a média populacional $\bar X$ com níveis de 90\%, 95\% e 99\%.

\vspace{0.5em}
Sabemos que cada amostra tem tamanho $n=800$ e, para $U(0,6)$, $\mu=3$. Desejamos
\[
P\bigl(L \le \bar X \le U\bigr) = \lambda,
\]
com $\lambda\in\{0.90,0.95,0.99\}$.

\vspace{0.7em}
Pelo Teorema do Limite Central:
\[
\bar X \;\overset{\text{aprox.}}{\sim}\;\mathcal N\!\bigl(\mu,\;\sigma/\sqrt{n}\bigr).
\]
Como $n$ é grande, usamos $\sigma\approx s$. Normalizando:
\[
Z = \frac{\bar X - \mu}{s/\sqrt{n}}
\quad\Longrightarrow\quad
Z \sim \mathcal N(0,1).
\]
\end{frame}

% Frame 2: Construção do IC
\begin{frame}{b) Intervalos de Confiança (2/3)}
Após a normalização, temos
\[
P(Z_1 \le Z \le Z_2) = \lambda,
\]
onde
\[
P(Z_1 \le Z)=P\bigl(L \le \bar X\bigr)=\tfrac{1-\lambda}{2},
\quad
P(Z \le Z_2)=P\bigl(\bar X \le U\bigr)=\tfrac{1-\lambda}{2}.
\]
Como $Z_2=-Z_1=\rho$, segue
\[
P(-\rho \le Z \le \rho)=\lambda
\;\Longrightarrow\;
P\!\Bigl(\bar X - \rho\,\tfrac{s}{\sqrt n} \le \mu \le \bar X + \rho\,\tfrac{s}{\sqrt n}\Bigr).
\]
Assim, o intervalo de confiança é
\[
IC = \Bigl(\bar X - \rho\,\tfrac{s}{\sqrt n},\;\bar X + \rho\,\tfrac{s}{\sqrt n}\Bigr),
\]
com
\[
\rho =
\begin{cases}
1.645, & \lambda=0.90,\\
1.960, & \lambda=0.95,\\
2.576, & \lambda=0.99.
\end{cases}
\]
\end{frame}

% Frame 3: Código
\begin{frame}[fragile]{b) Intervalos de Confiança (3/3)}
\lstinputlisting[
  language=Scilab,
  basicstyle=\ttfamily\scriptsize,
  caption={Cálculo dos intervalos de confiança}
]{../q04/questao04_calculo_ic.sci}
\end{frame}

\begin{frame}[fragile]{c) Cobertura dos Intervalos}
\textbf{c)} Verifique o percentual dos intervalos estatísticos com grau de confiança de 90\% que realmente contem o valor real da média populacional de $\bar X$. E para os graus de confiança de 95\% e 99\%?

\vspace{0.5em}

\lstinputlisting[
  language=Scilab,
  basicstyle=\ttfamily\scriptsize,
  caption={Verificação da proporção de cobertura (parte 1)}
]{../q04/questao04_percentual_contem_media.sci}
\end{frame}

\begin{frame}[fragile]{Código: Verificação da Proporção de Cobertura (Parte 2)}
\lstinputlisting[
  language=Scilab,
  basicstyle=\ttfamily\scriptsize,
  caption={Verificação da proporção de cobertura (parte 2)}
]{../q04/questao04_percentual_contem_media_2.sci}
\end{frame}