\section{Questão 4}

% -------------------------------
\begin{frame}{Descrição da Questão}
Considere que $X_1, X_2, \ldots, X_n$ são variáveis aleatórias independentes e identicamente distribuídas (i.i.d.) com distribuição uniforme contínua no intervalo $[0,6]$. Define-se a média amostral como:

\[
\bar{X} = \frac{1}{n} \sum_{i=1}^{n} X_i
\]

O objetivo é analisar o comportamento da média amostral com base em simulações e aplicar conceitos estatísticos, como o Teorema Central do Limite e Intervalos de Confiança.
\end{frame}

% -------------------------------
\begin{frame}[fragile]{Letra a) Geração das Amostras}
Geramos 1200 amostras de tamanho $n = 800$ da distribuição $U(0,6)$ utilizando a função \texttt{grand} do Scilab.

\vspace{0.3em}
\lstinputlisting[
    language=Scilab,
    basicstyle=\ttfamily\scriptsize,
    caption={questao04\_amostras\_ic.sci}
]{../q04/questao04_amostras_ic.sci}

\vspace{0.3em}
Como $U(0,6)$ tem média $\mu = 3$ e variância $\sigma^2 = 3$, o Teorema Central do Limite garante que $\bar{X}$ é aproximadamente normal para grandes $n$.
\end{frame}

% -------------------------------
\begin{frame}[fragile]{Letra b) Intervalos de Confiança}
Utilizando o Teorema Central do Limite e o desvio padrão amostral $s$, construímos intervalos de confiança com níveis de 90\%, 95\% e 99\%:

\[
IC = \left( \bar{X} \pm z_\alpha \cdot \frac{s}{\sqrt{n}} \right)
\]

\vspace{0.3em}
\lstinputlisting[
    language=Scilab,
    basicstyle=\ttfamily\scriptsize,
    caption={questao04\_calculo\_ic.sci}
]{../q04/questao04_calculo_ic.sci}

\vspace{0.3em}
Com valores:
\begin{itemize}
  \item $z_{0.90} = 1.645$
  \item $z_{0.95} = 1.960$
  \item $z_{0.99} = 2.576$
\end{itemize}
\end{frame}

% -------------------------------
\begin{frame}[fragile]{Letra c) Proporção de Intervalos que Contêm $\mu = 3$}
Verificamos a proporção de intervalos que efetivamente contêm o valor da média verdadeira.

\vspace{0.3em}
\lstinputlisting[
    language=Scilab,
    basicstyle=\ttfamily\scriptsize,
    caption={questao04\_percentual\_contem\_media.sci}
]{../q04/questao04_percentual_contem_media.sci}
\end{frame}

% -------------------------------
\begin{frame}{Conclusão}
\begin{itemize}
  \item A média das amostras geradas de $U(0,6)$ segue, aproximadamente, uma distribuição normal conforme previsto pelo Teorema Central do Limite.
  \item Os intervalos de confiança calculados mostram boa consistência com os níveis de confiança estipulados.
  \item Quanto maior o nível de confiança, mais largo o intervalo e maior a probabilidade de conter a média populacional.
\end{itemize}
\end{frame}
